
\begin{article}
\begin{center}

\large{
МИНИСТЕРСТВО ОБРАЗОВАНИЯ И НАУКИ РОССИЙСКОЙ ФЕДЕРАЦИИ}
\Large{Федеральное государственное автономное образовательное \\ 
учреждение высшего образования}

\Large{\textbf{
"Южно-Уральский государственный университет \\ 
(национальный исследовательский университет)" \\ 
Высшая школа электроники и компьютерных наук \\ 
Кафедра системного программирования}}\\

\vspace{0.5cm}

\begin{tabular}{ll}
\hspace{9cm} & УТВЕРЖДАЮ \\
& Зав. кафедрой СП \\
& \underline{\hspace{3cm}}Л.Б. Соколинский \\
& <<\underline{\hspace{0.5cm}}>>\underline{\hspace{3cm}}2018
\end{tabular}\

\vspace{1cm}

\Large{\textbf{
ЗАДАНИЕ \\ 
на выполнение курсовой работы} \\
по дисциплине <<Программная инженерия>> \\
студенту группы КЭ-301 \\
Бурулеву Антону Александровичу, \\
обучающемуся по направлению \\
02.03.02 «Фундаментальная информатика и информационные технологии» \\
}

\vspace{0.5cm}


\end{center}

\Large{
\begin{enumerate}
    \item[\textbf{1.}] \textbf{Тема работы}\\
    Разработка игры <<Монополия>> с использованием нейросетевых технологий
    \item[\textbf{2.}] \textbf{Срок сдачи студентом законченной работы:} 31.05.2018 г.
    \item[\textbf{3.}] \textbf{Исходные данные к работе}
    \begin{enumerate}
        \item [1.] Tariq Rashid. Make your own neural network, 2015. -240 p.
        \item [2.] Mnih, Volodymyr, Kavukcuoglu, Koray, Silver, David, Graves, Alex, Antonoglou, Ioannis, Wierstra, Daan, and Riedmiller, Martin. Playing atari with deep reinforcement learning // NIPS Deep Learning Workshop, 2013., pp. 1-9.
        \item [3.] Стюарт Р., Питер Н. Искусственный интеллект: Современный подход // Интеллектуальные агенты. 2-е издание. М.: Вильямс, 2006. С. 75-109.
    \end{enumerate}
    \item[\textbf{4.}] \textbf{Перечень подлежащих разработке вопросов}
    \begin{enumerate}
        \item [1.] Изучение методов обучения нейронных сетей
        \item [2.] Обзор используемых технологий
        \item [3.] Разработка игры <<Монополия>>
        \item [4.] Реализация нейронной сети для разработанной игры
        \item [5.] Тестирование полученной программы
    \end{enumerate}
\end{enumerate}
}
\textbf{Дата выдачи задания:} «9» февраля 2018 г.\\

\begin{onehalfspace}
\Large
\begin{tabular}{l{0.5\linewidth}l{0.5\linewidth}}
Научный руководитель \hspace*{5cm}& М.С. Тимченко \\
Преподаватель кафедры СП, \hspace{2cm}& \\
НИУ ЮУрГУ \hspace{2cm}& \\
\end{tabular}\
\end{onehalfspace}
\vspace{1cm}
\begin{onehalfspace}
\Large
\begin{tabular}{l{0.5\linewidth}l{0.5\linewidth}}
Задание принял к исполнению \hspace*{4.5cm}& А.А. Бурулев \\
\end{tabular}\
\end{onehalfspace}

\end{article} 
