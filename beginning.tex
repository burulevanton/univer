\newpage
\section*{Введение}
\addcontentsline{toc}{section}{Введение}
\textbf{Актуальность темы}

На данный момент основными способами разработки искусственного интеллекта для различных игр является конечный автомат и дерево вариантов [10]. С развитием науки и увеличением мощности вычислительной техники стало популярным машинное обучение, в частности нейронные сети. Стандартные способы обучения плохо подходят для разработки искусственного интеллекта из-за необходимости составлять обучающую выборку, что является трудоёмкой задачей для больших игр. Для таких задач отлично подходит метод обучения с подкреплением. 

Создание самообучающихся алгоритмов на основе Reinforcement Learning (Обучение с подкреплением) является актуальной задачей в современном мире, поскольку данный метод работает без заранее подготовленных ответов. Эта способность данного метода открывает возможности для решения нового вида задач. В частности, этот метод используется для обучения компьютеров прохождению игр. Создание подобных самообучающихся алгоритмов имеет важное практическое и научное значение, поскольку такие алгоритмы могут послужить основой для дальнейших исследований и совершенствований для создания алгоритмов, решающих реальные задачи. 

\textbf{Цели и задачи работы}

Целью данной работы является разработка игры «Монополия» с использованием нейросетевых технологий.

Основные задачами работы следующие:
\begin{spacing}{0.9}
\begin{enumerate}
    \item Изучение методов обучения нейронных сетей
    \item Обзор используемых технологий
    \item Разработка игры «Монополия»
    \item Реализация нейронной сети для разработанной игры
    \item Тестирование полученной программы
\end{enumerate}
\end{spacing}