\newpage
\setlength{\parskip}{0cm}
\begin{Large}
\addcontentsline{toc}{section}{Литература}
\begin{thebibliography}{99}
\bibitem{1} Коваленко А.С., Ковалевский В.Н. Построение логики игрового противника путём использования машинного обучения // Новые задачи технических наук и пути их решения. Сборник статей Международной научно-практической конференции, 2017.  С. 140-144.
\bibitem{2} Латкин И.И. Обзор возможностей TensorFlow для решения задач машинного обучения // Молодежный научно-технический вестник, 2016. 
\bibitem{3} Созыкин А. В. Обзор методов обучения глубоких нейронных сетей // Вестник Южно-Уральского государственного университета. Серия: Вычислительная математика и информатика, 2017. С. 28-49
\bibitem{4} Сотников И.Ю., Григорьева И.В. Адаптивное поведение программных агентов в мультиагентной компьютерной игре // Вестник Кемеровского государственного универститета, 2014. С. 65-71
\bibitem{5} Шампандар А. Искусственный интеллект в компьютерных играх. М.: Вильямс, 2007 - 768 с
\bibitem{6} Mnih, Volodymyr, Kavukcuoglu, Koray, Silver, David, Graves, Alex, Antonoglou, Ioannis, Wierstra, Daan, and Riedmiller, Martin. Playing atari with deep reinforcement learning // NIPS Deep Learning Workshop. 2013. pp. 1-9
\bibitem{7} Robert B Ash, Richard L Bishop, ‘Monopoly as a markov process’, Mathematics Magazine, 45(1), 1972, pp. 26–29
\bibitem{8} Tariq Rashid. Make your own neural network, 2015. 240 p
\bibitem{9} Официальный сайт TensorFlow. [Электронный ресурс] URL: https://www.tensorflow.org (дата обращения 21.03.2018).
\bibitem{10} Keras: The Python Deep Learning library [Электронный ресурс] URL: http://keras.io (дата обращения 21.03.2018)
\end{thebibliography}
\end{Large}